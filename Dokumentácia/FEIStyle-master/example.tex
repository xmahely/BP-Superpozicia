\documentclass[bp]{FEIstyle}

\makeglossaries

\newglossaryentry{latex}
{
        name=latex,
        description={Is a mark up language specially suited for 
scientific documents}
}

\newglossaryentry{maths}
{
        name=mathematics,
        description={Mathematics is what mathematicians do}
}

\newglossaryentry{formula}
{
        name=formula,
        description={A mathematical expression}
}

\newacronym{gcd}{GCD}{Greatest Common Divisor}

\newacronym{lcm}{LCM}{Least Common Multiple}

\begin{document}

The \Gls{latex} typesetting markup language is specially suitable 
for documents that include \gls{maths}. \Glspl{formula} are 
rendered properly an easily once one gets used to the commands.

Given a set of numbers, there are elementary methods to compute 
its \acrlong{gcd}, which is abbreviated \acrshort{gcd}. This 
process is similar to that used for the \acrfull{lcm}.

\clearpage

\printglossary[type=\acronymtype]

\printglossary

\end{document}

% \FEIauthor{Ján Srnec}
% \FEItitle{Ukážkový \LaTeX\ dokument s dlhým názvom}
% \FEItitleEn{Example \LaTeX\ document with long title}
% \FEIkeywords{kľúčové slovo1, kľúčové slovo2, kľúčové slovo3}
% \FEIkeywordsEn{keyword1, keyword2, keyword3}
% \FEIregNr{FEI-xxxx-xxxx}
% \FEIsupervisor{Mgr. Ing. Peter Párker, PhD.}
% \FEIconsultant{Ing. John Doe}

% \FEIglossaries{includes/glossary}
% \bibliography{includes/bibliography.bib}

% \iffalse \newacronym{sw}{SW}{Star Wars}
\newacronym{hw}{HW}{Halo Wars}
\newacronym{cdma}{CDMA}{Code Division Multiple Access} 
\newacronym{gsm}{GSM}{Global System for Mobile communication}

\newacronym{bnn}{BNN}{Biologická neurónová sieť}
\newacronym{ann}{ANN}{Umelá neurónová sieť}
\newacronym{rnn}{RNN}{Rekurentná umelá neurónová sieť}
 \fi

% \begin{document}
% \frontmatter

% \FEIpdfInfo
% \FEIcover
% \FEItitlePage
% % \FEIassignment{includes/assignment.jpeg}
% % \FEIassignment{includes/assignment_part2.jpeg}
% \FEIabstract{includes/abstract}
% \FEIabstractEn{includes/abstractEN}
% \FEIthanks{includes/thanks}
% \FEIcontent
% \FEIlistOfFiguresAndTables
% \FEIlistOfGlossaries
% \FEIlistOfAlgorithms
% \FEIlistOfListings

% \mainmatter

% \FEIintroduction{includes/introduction}
% \FEIcore{includes/core}
% \FEIconclusion{includes/conclusion}
% % \FEIresume{includes/resume} % Use only iff document is in english language

% % bibliography should use UTF-8 accents (write as is ľščťžýáí...) NOT converted by BibDesk
% % http://tex.stackexchange.com/questions/57743/how-to-write-%C3%A4-and-other-umlauts-and-accented-letters-in-bibliography
% \FEIbibliography %includes/bibliography.bib

% \backmatter

% \FEIlistOfAppendix
% % \FEIappendix{Štruktúra elektronického nosiča\label{att:A}}{includes/attachmentA}
% % \FEIappendix{Algoritmus\label{att:B}}{includes/attachmentB}
% % \FEIappendix{Výpis sublime\label{att:C}}{includes/attachmentC}
% \end{document}