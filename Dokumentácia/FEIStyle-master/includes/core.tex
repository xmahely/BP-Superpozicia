
\section{Recitácia}
Citujem všetky zdroje v 
\textbf{bibliography.bib}, 
% \cite{t00, t01, t02, t03, kniha, kniha2, kniha3, small, big, cs, koll, kap, tug, knuth, zbornik, prispevok}. 
\newline 

\section{Biologické neurónové siete}
 %ZDROJ
 %https://www.researchgate.net/profile/Vladimir-Kvasnicka/publication/266883364_Uvod_do_teorie_neuronovych_sieti/links/55291dcf0cf29b22c9bd20e7/Uvod-do-teorie-neuronovych-sieti.pdf
%https://www.fmed.uniba.sk/fileadmin/lf/sluzby/akademicka_kniznica/PDF/Elektronicke_knihy_LF_UK/Nervovy_system_I.pdf
%http://dai.fmph.uniba.sk/courses/NN/haykin.neural-networks.3ed.2009.pdf
Treba includnuť aj skratku ? do titulu alebo v texte ? \acrlong{bnn} Môžu sa používať skratky z angličitny alebo mám bnn označiť ako bns ? 
\newline
\newline
Nervový systém je tvorený nervovým tkanivom, ktoré sa vyznačuje schopnosťou prijímať rôzne informácie, spracovávať ich a efektívne na ne reagovať.
\newline
Nervové tkanivá pozostávajú z nervových buniek tzv. neurónov a gliových buniek tzv. neuroglií.
Neróny sú zodpovedné za tvorbu, spracovania a prenos signálova a neuroglie sú zodpovedné za podporu neurónov.
Prenos signálu je realizovaný formou zmeny membránového potenciálu neurónov.
%ZDROJ
%https://www.verywellmind.com/how-many-neurons-are-in-the-brain-2794889
\newline
Odhadovaný počet neúronov v ľudskom mozgu sa dlho odhadoval na 100 miliárd. Podľa najnovšieho výskumu sa však číslo približuje k 86 miliardám.
Na porovnanie, napríklad taká vínna muška má v mozgu 100 tisíc neúrovov, šimpanz ich má 7 miliárd a slon v priemere 257 miliárd.
\subsection{Neuróny}
Neurón je základnou funkčou jednotkou nervového systému. Existuje viac ako 50 typov, ktoré sa síce
od seba funkcionálne líšia, ale zdieľajú spoločné vlastnosti.
\newline
Neurón sa skladá z: 
\begin{itemize}
  \item bunkového tela
  \item dendritov, čo sú krátke výbežky tela, ktoré prijímajú signály
  \item axónu, ktorý prenáša nervové vzruchy k axónovým zakončeniam
\end{itemize}
\newline
--to do: obrázok neurónu
\subsection{Synapsy}
Synapsa je funkčné spojenie v mieste tesného kontaktu neurónu s ďalším neurónom.
Rozoznávame 3 typy: chemické a elektrické a zmiešané.
% ZDROJ
% https://cs.weblogographic.com/difference-between-chemical
\newline
Chemická synapsa
\newline
Elektrická synapsa
\newline
Zmiešaná synapsa

\subsection{Prenos signálov}
%\newline


\section{Umelé neurónové siete}
Treba includnuť aj skratku ? do titulu alebo v texte ? \acrlong{ann}
\subsection{Umelý neurón}
Definícia
\newline
Komponenty - vstupy + bias, váhy -> vážený priemer, aktivačná funkcia, výstup
\newline
presnejší popis aktivačných funkcii ? 
\subsubsection{Aktivačné funckie}

\subsection{Topológia}
\subsubsection{Perceptrón}
\subsubsection{Dopredné siete}
\subsubsection{Rekurentné siete}
\subsubsection{Kohonenove siete}
lattice network - mriežkované siete
\subsection{Učenie siete}

\section{Jadro}
% Titi bude asi dobre na \acrlong{hw}
% asfasfasfasfsa \acrlong{HW} asdafaf
% asgdshffgad fdsg sg \acrfull{CDMA} asfa fasfas,v fdfds
\subsection{Analýza problému}
V časti Analýza problému autor uvádza súčasný stav riešenej problematiky doma i v zahraničí, dostupné informácie a poznatky týkajúce sa danej témy. Zdrojom pre spracovanie sú aktuálne publikované práce domácich a zahraničných autorov. Základné definície a formalizmy potrebné na riešenie problematiky.
\newline \newline
Úloha:  navrhnúť a implementovať jednoduchý systém, ktorý bude zbierať produkty z rôznych zdrojov, na základe logiky bude potrebné urobiť produktový zoznam a disponentský model pre skupinu používateľov, vytvoriť jednoduchý rozhodovací systém, ktorý podľa rôznych parametrov a zdrojov bude na klientských produktoch pridávať príznaky.
\newline \newline
Analýza návrhu: systém by bolo možné implementovať rôznymi spôsobmi.
\newline \newline
Zdanlivo najjednoduchším prístupom by bol bolo napísať jednoduchý program, ktorý by v cykle čítal všetky záznamy a na základne jednoznačného identifikátora by ich navzájom porovnával a následne vkladal záznamy do novej tabuľky s alebo bez príznaku. Muselo by sa stanoviť, ako porovnávať záznamy v prípade chýbajúceho jednoznačného identifikátoru. Potom by pravdepodobne do porovnávacích podmienok muselo vstupovať viacero rôznych  parametrov v závislosti od zdrojových dát a existencie jednoznačného identifikátoru.
Takéto riešenie by však nebolo najvhodnejšie, veľmi ľakho sa totiž môže stať, že sa v takejto analýze vynechá jeden krajný prípad, ktorý by mohol spôsobiť stovky a tisícky zle vyhodnotených príznakov v záznamoch. Takto vyhodnotené príznaky by sa hladali ťažko, keďže krajný prípad nebol definovaný v analýze, a ak by sa v testovaje fáze našla chyba, musela by sa znova prepisovať analytický časť a upravovať už existujúce podmienky.
\newline \newline
%file:///E:/Filmy/Spy%20Hard%20(1996)%20[WEBRip]%20[1080p]%20[YTS.AM]/IGAEM_Improved_Genetic_Algorithm_based_Entity_Matc.pdf
Ďalším spôsobom vyhotovenia takéhoto programu by bol genetický algoritmus. Ten vychádza z Darwinovej evolučnej teórie. Algoritmus by mohol vyzerať nasledovne: 
Najskôr by bolo potrebné inicializovať populáciu o veľkosti x chromozónov. Chromozón je jedinec, ktorý predstavuje riešenie problému. Číselná hodnota, ktorá bude predstavovať vhodnosť/správnosť jedinca je priradená každému chromozónu. Na základe tejto hodnoty sa vyberú najvhodnejšie jedince na kríženie a mutáciu. V závislosti od vhodných kandidátov na správne riešenie sa vynegeruje nová populácia  a celý proces sa opakuje niekoľkokrát. Výsledkom by mal byť najvhodnejší chromozón, ktorý predstavuje korektné riešenie.
\newline \newline
Implementácia pomocou neurónovej siete. 
Pri vhodne upravených dátach by bolo možné záznamy z rôznych zdrojov medzi sebou porovnávať.
Najskôj by museli byť zostrojené trénovacie dáta s manuálne pridaným správnym výsledkom, na ktorých by sa sieť natrénovala.
Na ostrých dátach, bez správneho výsledku, by potom sieť vyhodnotila na koľko percent sú záznamy duplicitné.
A podľa stanovenej podmienky by k hlavnému záznamu partnera pridala/nepridala príznak duplicitného záznamu z iného zdroja.

\subsection{Opis riešenia}
Časť Opis riešenia jasne, výstižne a presne charakterizuje predmet riešenia. Súčasťou sú aj rozpracované čiastkové ciele, ktoré podmieňujú dosiahnutie hlavného cieľa. Ak je práca implementačná, tak jej súčasťou musí byť aj softvérová špecifikácia požiadaviek, návrh, implementácia, overenie riešenia. Treba podľa možností vychádzať zo známych prístupov. Táto časť práce závisí od konkrétneho zadania. Je dôležité prezentovať návrhové rozhodnutia, alternatívy, ktoré sa zvažovali pri riešení a samotný návrh riešenia zadaného problému. Štruktúra textu by mala vychádzať zo zadanej úlohy, ktorá sa rieši. Najmä v tejto časti študent preukazuje svoj originálny prístup k riešeniu problémov a kritické myslenie.
\newline
Súčasťou môže byť metodika práce a metódy skúmania, ktoré spravidla obsahujú: 
a) charakteristiku objektu skúmania 
b) pracovné postupy 
c) spôsob získavania údajov a ich zdroje 
d) použité metódy ich vyhodnotenia a interpretácie výsledkov 
Implementácia musí byť otestovaná. Výsledok musí byť porovnaný s inými riešeniami.
\newline \newline
todo: tu treba dopísať riešenie problému 
\newline
výber jazyka, knižník a pod má byť v tejto časti ? 
\newline
rišenie prostredníctvom rekurentnej neúronovej siete lebo ..
\newline
popísať ako treba upraviť vstupné dáta - normalizácia 
\newline
+ do každej tabuľky označiť prioritu zdroja ? možno nebude treba keď budem podľa priority joinovať 
ostatnéta tabulky
\newline
asi na začiatku bude treba vytvoriť dáta na trénovanie a manuálne im označiť match na  hodnotu 0/1
\newline
nahranie tabulky sporky do tabulky suporpozicia - čo bude výstup programu 
\newline
k sporke najoinovať každý zdroj, to by mal byť koniec pre upravu dat
\newline
popísať ako má vyzerať výsledná najoinovaná tabuľka, jeden záznam, druhý záznam a match
\newline
natrénovať model, testovanie, validácia - rozdeliť dáta 
\newline
rnn na nelablovaných dátach, výstupom bude tabulka s match hodnotou 
\newline
v závislosti od percenta potom budem pridávať príznak k hlavnému záznamu do superpozicie


\subsection{Zhodnotenie}
Výsledky (vlastné postoje alebo vlastné riešenie vecných problémov), ku ktorým autor dospel, sa musia logicky usporiadať a pri popisovaní sa musia dostatočne zhodnotiť. Zároveň sa komentujú všetky skutočnosti a poznatky v konfrontácii s výsledkami iných autorov. Ak je to vhodné, výsledky práce a diskusia môžu tvoriť samostatné časti ZP.





